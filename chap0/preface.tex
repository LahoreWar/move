\label{chap:preface}
\begin{table}[htbp]
	\newcommand{\tabincell}[2]{\begin{tabular}{@{}#1@{}}#2\end{tabular}} %换行指令
	\centering
	\caption{名词列表 \label{tab:0_1}}
	\renewcommand\arraystretch{1.0}	%设置表格内行间距
	\setlength{\tabcolsep}{8mm}{
	\begin{tabular}{llll}
		\toprule 
		 名词(缩略词)   && 定义 \\
		 \midrule
		 stance phase   && 支撑期 \\
		\midrule
		swing phase && 摆动期 \\

		\bottomrule  

	\end{tabular}}
\end{table}%





\begin{table}[htbp]
	\newcommand{\tabincell}[2]{\begin{tabular}{@{}#1@{}}#2\end{tabular}} %换行指令
	\centering
	\caption{生物学和生理学的术语表 \label{tab:0_2}}
	\renewcommand\arraystretch{1.0}	%设置表格内行间距
	\setlength{\tabcolsep}{8mm}{
		\begin{tabular}{ll}
			\toprule 
			术语   & 用法和同义词 \\
			\midrule
			先进   & 与祖先状况不同  \\
			\midrule
			类比     & 具有共同功能的两个或多个物种的结构或行为   \\
			\midrule
			类人猿     & \makecell{一组灵长类动物,包括所有现代猴子、猿和人类,\\以及祖先类人猿的已灭绝后代}    \\
			\midrule
			注意力      &对可用信息、感官或助记符的子集的增强处理   \\
			\midrule
			狭鼻类动物       & 一组灵长类动物,包括旧世界的猴子、猿和人类   \\
			\midrule
			选择       &在备选方案中选择目标或行动   \\
			\midrule
			结合       & 代表性元素的组合   \\
			\midrule
			当前上下文       &感官输入和最近发生的事件   \\
			\midrule
			决策      &对世界的看法   \\
			\midrule
			情景记忆       &回忆事件,暗示意识   \\
			\midrule
			事件      &特定时间和地点的情境、目标、行动和结果的一次性结合   \\
			\midrule
			外部指导       &基于外部感官输入的行为  \\
			\midrule
			目标       &作为动作目标的物体或地方   \\
			\midrule
			习惯      &过度训练的结果,在不参考预测结果的情况下对刺激产生响应   \\
			\midrule
			类人猿亚目      &一组灵长类动物,包括眼镜猴和类人猿   \\
			\midrule
			同源性      &由于共同祖先的遗传而出现在两个或多个物种中的特征   \\
			\midrule
			“内部”指导      &当没有感官输入提示行为时   \\
			\midrule
			记忆       &存储信息   \\
			\midrule
			需要      &食物和液体等生物学要求; 同义词:动力、动机   \\
			\midrule
			结果      &刺激或行为产生的好处或伤害   \\
			\midrule
			优势响应、行为      &天生的、习惯性的或条件反射   \\
			\midrule
			原始     &类似于祖先的情况   \\
			\midrule
			前瞻、前瞻记忆、前瞻编码       &短期记忆中目标的表示   \\
			\midrule
			强化       &作为反馈的结果   \\
			\midrule
			再表示       &基于其他低阶表示的神经表示   \\
			\midrule
			响应       &依赖于与刺激或结果的条件关联的行动   \\
			\midrule
			奖励       &有益的结果   \\
			\midrule
			规则      &行为输入输出算法   \\
			\midrule
			符号      &小于整个对象但大于基本感官特征的非空间提示   \\
			\midrule
			策略      &(1) 一个问题的两个或多个解决方案中的一个; (2) 部分解决问题   \\
			\midrule
			值      &成本或收益的程度   \\
			\bottomrule  
			
	\end{tabular}}
\end{table}%

