\label{chap:preface}
\begin{table}[htbp]
	\newcommand{\tabincell}[2]{\begin{tabular}{@{}#1@{}}#2\end{tabular}} %换行指令
	\centering
	\caption{名词列表 \label{tab:0_1}}
	\renewcommand\arraystretch{1.0}	%设置表格内行间距
	\setlength{\tabcolsep}{8mm}{
	\begin{tabular}{llll}
		\toprule 
		 名词(缩略词)   && 定义 \\
		 
		 \midrule
		 cost of transport && 单位距离能耗 \\
		 
		 \midrule
		 dorsiflexion && 背屈 \\
		 
		 \midrule
		 extensor digitorum longus && 伸趾长肌 \\
		 
		 \midrule
		 foot progression angle && 足前进角 \\
		 
		 \midrule
		 gastrocnemius && \href{https://baike.baidu.com/item/%E8%85%93%E8%82%A0%E8%82%8C}{腓肠肌} \\
		 
		 \midrule
		 gluteus maximus && 臀大肌 \\
		 
		 \midrule
		 hamstrings && \href{https://baike.baidu.com/item/hamstring/51109946}{腘绳肌腱} \\
		 
		 \midrule
		 Kat Steele && 凯特$\cdot$斯蒂尔 \\
		 
		 \midrule
		 Matthew Abbate && 马修$\cdot$阿贝特 \\
		 
		 \midrule
		 Metabolic power && 代谢功率 \\
		 
		 \midrule
		 Molly Seamans && 莫莉$\cdot$西曼斯 \\
		 
		 \midrule
		 pentapedal gait && 五足步态 \\
		 
		 \midrule
		 % zhi2
		 plantarflexion && \href{https://baike.baidu.com/item/%E8%84%9A%E5%BA%95%E5%BC%AF%E6%9B%B2}{跖屈} \\
		 
		 \midrule
		 rectus femoris   && \href{https://baike.baidu.com/item/%E8%82%A1%E7%9B%B4%E8%82%8C}{股直肌} \\
		 
		 \midrule
		 Silvia Blemker   && 西尔维亚$\cdot$布莱姆克尔 \\
		 
		 \midrule
		 soleus   && \href{https://baike.baidu.com/item/%E6%AF%94%E7%9B%AE%E9%B1%BC%E8%82%8C}{比目鱼肌} \\
		 
		 \midrule
		 stance phase   && 支撑阶段 \\
		 
		 \midrule
		 step length   && 步长 \\
		 
		 \midrule
		 step width   && 踏步宽度 \\
		 
		 \midrule
		 stride length   && 步幅 \\
		 
		 \midrule
		 swing phase && 摆动阶段 \\
		 
		 \midrule
		 tibialis anterior && \href{https://baike.baidu.com/item/%E8%83%AB%E9%AA%A8%E5%89%8D%E8%82%8C}{胫骨前肌} \\
		 
		 \midrule
		 tibialis posterior && 胫骨后肌 \\
		 
		 \midrule
		 toe-out angle && 足偏脚 \\
		 
		 \midrule
		 total knee replacement && \href{https://baike.baidu.com/item/%E5%85%A8%E8%86%9D%E5%85%B3%E8%8A%82%E7%BD%AE%E6%8D%A2%E6%9C%AF/15634686}{全膝关节置换术} \\
		 
		 % 股肌(Vasti)(除股直肌以外的其他三条股四头肌)起于股骨体,而臀大肌则止于股骨体的后侧面
		 \midrule
		 Vasti && 股肌 \\

		\bottomrule  

	\end{tabular}}
\end{table}%





\begin{table}[htbp]
	\newcommand{\tabincell}[2]{\begin{tabular}{@{}#1@{}}#2\end{tabular}} %换行指令
	\centering
	\caption{生物学和生理学的术语表 \label{tab:0_2}}
	\renewcommand\arraystretch{1.0}	%设置表格内行间距
	\setlength{\tabcolsep}{8mm}{
		\begin{tabular}{ll}
			\toprule 
			术语   & 用法和同义词 \\
			\midrule
			先进   & 与祖先状况不同  \\
			\midrule
			类比     & 具有共同功能的两个或多个物种的结构或行为   \\
			\midrule
			类人猿     & \makecell{一组灵长类动物,包括所有现代猴子、猿和人类,\\以及祖先类人猿的已灭绝后代}    \\
			\midrule
			注意力      &对可用信息、感官或助记符的子集的增强处理   \\
			\midrule
			狭鼻类动物       & 一组灵长类动物,包括旧世界的猴子、猿和人类   \\
			\midrule
			选择       &在备选方案中选择目标或行动   \\
			\midrule
			结合       & 代表性元素的组合   \\
			\midrule
			当前上下文       &感官输入和最近发生的事件   \\
			\midrule
			决策      &对世界的看法   \\
			\midrule
			情景记忆       &回忆事件,暗示意识   \\
			\midrule
			事件      &特定时间和地点的情境、目标、行动和结果的一次性结合   \\
			\midrule
			外部指导       &基于外部感官输入的行为  \\
			\midrule
			目标       &作为动作目标的物体或地方   \\
			\midrule
			习惯      &过度训练的结果,在不参考预测结果的情况下对刺激产生响应   \\
			\midrule
			类人猿亚目      &一组灵长类动物,包括眼镜猴和类人猿   \\
			\midrule
			同源性      &由于共同祖先的遗传而出现在两个或多个物种中的特征   \\
			\midrule
			“内部”指导      &当没有感官输入提示行为时   \\
			\midrule
			记忆       &存储信息   \\
			\midrule
			需要      &食物和液体等生物学要求; 同义词:动力、动机   \\
			\midrule
			结果      &刺激或行为产生的好处或伤害   \\
			\midrule
			优势响应、行为      &天生的、习惯性的或条件反射   \\
			\midrule
			原始     &类似于祖先的情况   \\
			\midrule
			前瞻、前瞻记忆、前瞻编码       &短期记忆中目标的表示   \\
			\midrule
			强化       &作为反馈的结果   \\
			\midrule
			再表示       &基于其他低阶表示的神经表示   \\
			\midrule
			响应       &依赖于与刺激或结果的条件关联的行动   \\
			\midrule
			奖励       &有益的结果   \\
			\midrule
			规则      &行为输入输出算法   \\
			\midrule
			符号      &小于整个对象但大于基本感官特征的非空间提示   \\
			\midrule
			策略      &(1) 一个问题的两个或多个解决方案中的一个; (2) 部分解决问题   \\
			\midrule
			值      &成本或收益的程度   \\
			\bottomrule  
			
	\end{tabular}}
\end{table}%

