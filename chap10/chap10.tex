\chapter{肌肉驱动模拟} \label{chap:chap10}


预测非常困难,
尤其是关于未来的预测。
\begin{flushright}
	——尼尔斯·玻尔
\end{flushright}

1983年大学毕业后,我的第一份工作是帮助小公司编写计算机辅助设计软件。
当时我在科罗拉多州的一家计算机工厂工作,这家工厂刚刚生产出一台功能强大的新型图形计算机。
在当时,“强大”意味着它每秒可以在小屏幕上画几条线。
如果没有图形软件,没有人会使用我们的图形计算机,所以我的工作就是帮助其他公司的工程师将他们的计算机辅助设计软件在我们新推出的计算机上运行。
我逐渐意识到,几乎所有未来的产品都将在计算机上设计。
在我申请研究生院的时候,我提议开发用于手术设计的计算机图形工具。


两年后,我进入研究生院,有幸加入了斯坦福大学设计组由费利克斯$\cdot$扎亚克领导的生物力学研究实验室。
扎亚克是一位热衷于理解运动控制的神经科学家,他几十年来一直在进行各种精妙的实验,测量跳跃等动作中的肌肉激活模式、地面反作用力以及关节运动。
他意识到,要将这些实验数据整合起来,全面理解运动过程中的肌肉功能,仅仅依靠专业的数据分析是不够的。


仅靠实验测量不足以了解运动过程中的肌肉动作,原因有二。
首先,像肌肉产生的力量这样重要的量通常无法在实验中测量。
其次,仅通过实验观察很难建立因果关系。
例如,可以在行走过程中测量地面反作用力(第~\ref{chap:chap2}~章),并将其用于估计身体重心的加速度。
然而,单靠地面反作用力测量几乎无法了解肌肉如何影响身体重心的加速度,从而无法了解肌肉如何影响行走过程中支撑身体重量和推动身体向前的关键任务。
可以分析 EMG 信号(第~\ref{chap:chap4}~章)来了解肌肉何时活跃,但不能揭示哪些身体运动是由每块肌肉的活动引起的。


我们需要一个新的框架来推进我们对运动过程中肌肉功能的理解。
这个框架必须揭示肌肉激活、肌肉力量、地面反作用力和身体运动之间的关系。


肌肉驱动的运动模拟提供了这一框架。
模拟可以估算肌肉力量,并揭示因果关系,例如行走过程中肌肉对地面反作用力的贡献。
我们还可以利用模拟来预测身体对疾病、手术或肌肉激活改变的反应。
这些能力使我们能够表征运动过程中肌肉的动作,并设计手术和辅助设备。


1985年,当我加入扎亚克的研究小组时,他和他的学生们正处于开发肌肉驱动模拟的前沿。
加入这个小组后,我开启了一段持续30多年的旅程,专注于创建肌肉驱动模拟并进行分析,以改善生物力学受损人群的运动能力。
大学毕业后,我从事的工作在两年内就开发出了用于工程产品的计算机辅助设计工具,而开发用于理解人体运动复杂性的计算机辅助设计工具却成了我毕生的挑战。


本章介绍了我在创建肌肉驱动模拟方面的一些经验。
本章首先阐述了为什么在没有模拟的情况下很难确定运动过程中肌肉的动作,以及为什么文献中充斥着关于肌肉功能的错误结论。
接下来,我们将讨论构建和分析肌肉驱动模拟以正确确定肌肉动作的 4 个阶段。
然后,本章介绍了我和同事开发的开源模拟软件,该软件旨在促进全球合作,让成千上万的研究人员能够构建和共享运动的计算机模拟。
我的目标是通过齐心协力推动这一领域的发展。


\section{理解运动过程中的肌肉动作是一项挑战}

基于肌肉几何形状、肌电图 (EMG) 测量和观察到的运动来推断其动作的实验方法无法正确解释肌肉如何驱动身体。
仅基于解剖学知识的分析常常会导致关于肌肉功能的错误结论。
例如,许多解剖学和生物力学文献将比目鱼肌描述为使踝关节跖屈的肌肉。
比目鱼肌确实会产生踝关节跖屈力矩,从而确实使踝关节跖屈,但该肌肉也能执行其他动作(图 10.1)。
这些动作源于一种称为动态耦合的效应。


动态耦合描述了一个身体节段的运动由于诱导力而影响另一个节段运动的现象。
如图 10.1 所示,比目鱼肌产生的力不仅会产生踝关节跖屈力矩,还会诱导全身节段间的力和关节加速度。
这些节段间的力的大小和方向取决于肌肉施加的力、肌肉的力臂、身体节段的质量和惯性以及身体的姿势。
在图 10.1 右侧的示例中,比目鱼肌产生的力使小腿产生逆时针的角加速度,这需要膝关节向上和向左加速。
大腿及其相邻节段的惯性抵抗了这种加速度,并在膝盖处产生节段间的力,这反过来又加速大腿,依此类推。
因此,尽管比目鱼肌只跨越踝关节,但它却加速了身体的所有关节。


在许多情况下,动态耦合产生的节段间力足够大,从而影响我们对肌肉动作的解读。
虽然远离肌肉的关节处的“肌肉诱导”加速度通常较小,但在附近关节处却可能很大。
例如,Felix Zajac 和 Michael Gordon (1989) 证明,在站立时,比目鱼肌使膝关节伸展的加速度甚至大于使踝关节跖屈的加速度。
此外,他们还指出,双关节肌肉可以诱导与它穿过的其中一个关节产生的力矩相反的关节加速度。
例如,虽然腓肠肌产生膝关节屈曲力矩和踝关节跖屈力矩,但它仍然可以诱导膝关节伸展加速度或踝关节背屈加速度(图 10.2)。
这些看似不协调的加速度在腓肠肌激活时是可能的,因为例如,它产生的膝关节屈曲力矩引起的膝关节屈曲加速度可能会被它产生的踝关节跖屈力矩引起的膝关节伸展加速度所掩盖。
许多生物力学研究在解释肌肉动作时忽略了动态耦合,并得出了错误的结论。
对于由数十个身体节段、关节和肌肉组成的肌肉骨骼系统来说,推断运动过程中肌肉的动作是一项挑战。
需要肌肉驱动的模拟来应对这一挑战。


你可能想知道肌肉是否真的会产生与施加力矩方向相反的加速度。
我以前也曾怀疑过。
史蒂夫$\cdot$皮亚扎是我实验室的一名学生,他创建了肌肉驱动的行走摆动阶段模拟(Piazza and Delp, 1996)。
他的模拟表明,在某些情况下,腘绳肌会产生髋屈曲加速度。
回想一下,腘绳肌在髋后交叉,因此,许多科学家认为这些肌肉总是会产生髋关节伸展。
运动方程分析证实了髋屈曲确实可能产生,但我们的临床同事对此表示怀疑。
尤其是杰奎琳$\cdot$佩里,一位世界领先的肌肉和步态专家,也是我的科学偶像之一,她不相信我们的结果,想要更多证据。
因此,史蒂夫制造了“说服器”,这是一种简单的装置,类似于一条腿,在腘绳肌所在的位置有一根金属丝。
当我们在适当的条件下拉动腘绳肌的金属丝时,髋关节会轻微弯曲。
史蒂夫和我都深信不疑,我们的临床同事,包括佩里博士,也深信不疑。


\section{创建肌肉驱动的模拟}

牛顿运动定律的方程表征了人体的动力学。
我们可以通过求解这些方程来预测人体的运动方式,这个过程被称为动态模拟。
“肌肉驱动”的动态模拟可以预测肌肉产生的力量在行走和跑步等运动过程中如何影响身体各个部位的运动。


开发、测试和分析肌肉驱动模拟的过程包括 4 个阶段(图 10.3)。
在第 1 阶段,您将创建一个计算模型,该模型能够以足够的精度描述肌肉骨骼系统的动态行为,以回答您的研究问题。
如果其他人已经创建并分享了适合您研究的模型,您可以跳过这个繁琐的步骤。
第 2 阶段涉及计算一组肌肉激励,当将这些激励应用于模型时,会生成感兴趣运动的模拟。
第 3 阶段通过将模拟结果与实验测量结果进行比较,确认模拟充分代表了感兴趣的运动。
在第 4 阶段,您将分析模拟以回答您的研究问题。
我们将在接下来的章节中探讨每个阶段。


\section{第一阶段:肌肉骨骼系统动力学建模}

肌肉骨骼动力学模型使我们能够计算由每种肌肉力量引起的运动。
我们四阶段流程的第一阶段是使用描述肌肉激活动力学、肌肉肌腱收缩动力学、肌肉骨骼几何结构和骨骼动力学的方程(图 10.4)来创建肌肉骨骼系统模型。
这些方程表征了肌肉骨骼系统响应肌肉刺激时的时间依赖性行为。


正如我们在第~\ref{chap:chap4}~章中看到的,肌肉兴奋和激活之间的关系受运动单元动作电位和横桥循环的动态控制。
肌肉的激活 ($a$) 可以通过将其时间导数 ($\dot{a}$) 与电流激活和兴奋 ($u$) 关联来建模,如公式 4.1 所示。


肌肉激活是肌肉-肌腱收缩动力学模型的输入,肌肉-肌腱执行器(和)的长度和伸长速度也是如此。
如第~\ref{chap:chap5}~章所述,肌肉和肌腱的动力学受横桥形成的时间过程、肌动蛋白丝的滑动以及肌腱的动力学控制。
肌肉产生的力量 ($F^M$) 并通过其肌腱传递的力量 ($F^T$) 可以用四条无量纲曲线和五个肌肉特定参数来估算,如第~\ref{chap:chap5}~章所述。
当应用于骨骼时,肌肉力量会产生关于关节的力矩,如第~\ref{chap:chap6}~章所述。
肌肉产生的关节力矩导致关节和身体节段加速,从而产生运动。


可以使用身体运动方程来计算身体对肌肉力量和其他负荷的响应加速度:
\begin{equation}
	\underline{\ddot{q}} = M^{-1} (\underline{q})
			\{
				\underline{F}^G (\underline{q}) + 
				\underline{F}^C (q, \dot{q}) + 
				R(\underline{q}) \underline{F}^T (\underline{u}) + 
				\underline{F}^E (\underline{q}, \underline{\dot{q}})
			\}
\end{equation}


